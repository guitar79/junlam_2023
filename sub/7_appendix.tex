\section{부록: Arduino Code for Wave Maker}

\subsection*{move.ino}
\begin{minted}[frame=lines, linenos, fontsize=\scriptsize]
{arduino}
void vel(int V){
  Serial.println("limitCheck");
  if(limitCheck(V)){
    float delta = V/MaxSpeed;
    delta = min(1.0f,max(-1.0f,delta));
    rotatecontrol.overrideSpeed(delta);
  }
  else vel(0);
}
boolean limitCheck(int V){
  if(analogRead(lmtPin1)<10&&V>0){
    rotatecontrol.overrideSpeed(0);
    return false;
  }
  if(analogRead(lmtPin2)<10&&V<0){
    rotatecontrol.overrideSpeed(0);
    return false;
  }
  return true;
}

void initializeMotor(){
  int V = 5000;
  while(limitCheck(V)){
    vel(V);
  }
  vel(0);
  leftEnd = motor.getPosition();
  delay(300);
  while(limitCheck(-V)){
    vel(-V);
  }
  rightEnd = motor.getPosition();
  vel(0);
  MID = (leftEnd+rightEnd)/2;
  gotoMiddle();
  Serial.println(leftEnd);
  Serial.println(rightEnd);
}
void gotoMiddle(){
    while(motor.getPosition()<MID){
      vel(5000);
    }
    vel(0);
    while(motor.getPosition()>MID){
      vel(-5000);
    }
    vel(0);
}
\end{minted}


\subsection*{sin\_wave.ino}
\begin{minted}[frame=lines, linenos, fontsize=\scriptsize]
{arduino}
int cmtostep = 400;
int A = 15;
int N = 50;
float w = 0;

void sin_set(float k,int a){
  Serial.println("sin set called");
  waveset = true;
  A = a;
  A*=400;
  w = k*PI;
   while(motor.getPosition()<MID+A){
    vel(5000);
  }
  vel(0);
  delay(5000);
  timer = 0;
  intervalTimer = 0;
}
void sinWave(){
  if(intervalTimer>N){
    intervalTimer = 0;
    float n_speed = (sinmillis(timer+N)-sinmillis(timer))/((float)N/1000);
    vel(n_speed);
    Serial.println(n_speed);
  }
}
float sinmillis(float T){
  NUM(0);
  return A*cos(w*T/1000);
}
\end{minted}