\section{결론}
\subsection{결론}

조파기는 

1. 조파기의 성능 ->
$\omega$별 정확도?
우리의 가정(진폭, 각진동수 사이 관계)은 어떻게 되었나?
됨


\section{결론 및 제언}

\subsection{결론}
	
본 연구는 천체 망원경 모터 포커서 컨트롤러인 GS-touch를 개발하는 것이고 개발 및 시험 동작에 성공하였다.  GS-touch\는 $12 \textrm{V}$로 구동되며 2상 바이폴라 스테핑 모터가 장착된 모터 포커서를 구동할 수 있다. 


본 연구 결과물을 요약해 보면 다음과 같다. 

\begin{itemize}
	\item 아두이노 나노를 기반으로 2상 바이폴라 스테핑 모터 드라이버, OLED 디스플레이어, 온습도 센서 등이 연결된 모터 포커서 컨트롤러인 GS-touch의 하드웨어를 제작하여 구동에 성공하였다. 전원은 DC 12V를 사용하며, 4개의 버튼으로 메뉴를 선택하거나 모터를 정방향 역방향으로 회전시킬 수 있도록 설계하였다.
	\item GS-touch를 구동할 수 있는 펌웨어을 제작하였다. 펌웨어의 기능은 DRV8825를 이용하여 스테핑 모터를 제어할 수 있도록 하였고, DHT22를 이용하여 온도, 습도 값을 읽어들여 OLED 디스플레이어에 표현할 수 있도록 하였다. 또한 버튼 조작으로 OLED 디스플레이어에 표시되는 메뉴를 조정할 수 있도록 하였다. 
	\item  GS-touch 전용의 ASCOM 드라이버를 개발하여 GS-touch를 PC에 연결하여 ASCOM을 지원하는 MaximDL, FocusMAX 등의 소프트웨어로 구동하는데 성공하였다.
\end{itemize}

학생의 연구 활동을 지도하는 과정에서 학생은 다음과 같이 성장하는 모습을 보여주었다. 

첫째, 하나의 GS-touch 컨트롤러를 제작하기까지 Circuit maker, Fusion 360, Arduino IDE, Visual Studio 2017, ASCOM flatform 등의 사용법을 학생에게 지도하였으며, 학생은 그 과정에서 연구 수행 능력이 눈에 띠게 향상되는 모습을 보여 주었다. 

둘째, 학생은 크고 작은 문제점이 발생될 때마다 코딩으로 그 문제를 해결하는 과정에서 문제 해결 능력이 향상되었으며, 이러한 학생의 능력은 다른 연구를 수행할 때도 큰 도움이 될 것으로 생각된다.  

셋째, 제작한 GS-touch\를 구동하기 위해 MaximDL, FocusMAX 등의 소프트웨어로 천체 관측 경험을 하였으며, 관측 천문학적 소양을 기를 수 있었다. 

넷째, 보고서 작성 단계에서는 \LaTex 사용법을 지도 하였다. 최근 여러 학술지에서도 논문을 \LaTex\로 투고 받는 경우들이 있다. 학생은 이미 \LaTeX\을 능숙하게 사용할 줄 아는 등 연구자로서의 자질이 충분하다고 볼 수 있다. 

\subsection{제언}

현재 개발된 GS-touch는 초보 단계로서 개선할 수 있는 부분들이 아직 많이 있다. 다음 버전의 하드웨어 부분에서는 버튼 스위치를 좀더 내구성이 좋은 부품으로 교체하는 것이 좋을 것으로 생각된다. 또한 OLED 와 센서의 전원은 아두이노에서 출력되는 전원을 사용하여 외부 전원을 연결하지 않아도 메뉴 조작이 가능하도록 개선할 생각이다. GS-touch 다음 버전 제작시에 개선해야 할 내용을 정리하면 다음과 같다. 

\begin{description}[font=$\bullet$~\normalfont\scshape\color{red!50!black}]
	
	\item [Backleash 보정 기능 추가] 모터 포커서를 반대 반향으로 구동할 때 발생할 수 있는 backleash 값을 측정하여 보정할 수 수 있는 기능을 추가할 수 있다.
	
	\item [MCU 변경] ARDUINO NANO로 펌웨어를 개발할 때 메모리 용량이 작아 어려움을 겪었다. ARDUINO NANO보다 성능이 좋은 STM32L432KC 같은 개발 보드를 사용하되, 핀 배열만 맞춰 주면 현재 개발한 펌웨어를 그대로 사용할 수 있다.
	
	\item [Heating system 추가] 천체 관측시 렌즈에 이슬이 맺혀 관측에 어려움을 겪는 경우가 종종 있다. 이를 해결하기 위하여 날씨가 추운 날에는 모터가 얼어서 돌아가지 않거나 렌즈에 서리가 껴서 초점이 맞아도 맞지 않은 것으로 판단할 수 있다. 따라서 이를 예방하기 위하여 열선을 깔아서 DHT22에서 측정한 온도나 습도를 이용하여 열선의 동작을 제어할 수 있다. 
	
	\item [EEPROM 활용] EEPROM은 Arduino 내부에 저장된 비휘발성 메모리로, 컴퓨터의 ‘RAM’과 같은 역할을 하고 있다. 비휘발성이기 때문에 Arduino를 초기화하거나 껐다가 다시 켰더라도 정보를 저장하고 있다. 
	Arduino별로 한 EEPROM의 주소에 들어갈 수 있는 수의 크기가 달라진다. ARDUINO NANO는 4KB의 EEPROM을 지원하므로 0~255까지의 수를 한 번에 저장할 수 있다. 이렇게 저장할 수 있는 수가 작으므로, 여러 가지 주소를 활용하여 큰 수 또한 나타낼 수 있다.(수를 진법으로 바꾸는 과정과 유사함) 실제로 이를 기반으로 펌웨어를 제작하여 보았지만, 수가 약 32000 이상으로 넘어가는 상황에서는 갑자기 수가 이상하게 커지는 오류가 발견되었고, 이를 고쳐야 할 것이다.
	
	\item [모터 연결 상태 체크 기능 추가] 모터의 연결 상태는 펌웨어를 실행하는 데 아주 중요하다. 만약 펌웨어가 실행되는 도중에 모터가 연결되지 않으면, 스위치를 움직였을 때 스위치의 숫자는 움직이지만, 모터는 움직이지 않아 결과적으로 숫자의 오류를 불러일으킨다. 또한, 모터를 펌웨어가 실행되는 도중에 연결선을 뽑으면 펌웨어에 에러가 일어나는데, 이 경우 다시 모터를 꽂더라도 정상적으로 실행이 되지 않는다. 따라서 이런 여러 상황에 대하여 모터의 연결 상태를 대비한 에러 코드를 설정해야 숫자와 모터가 오차를 일으키는 일이 없을 것이다.
\end{description}

또한 GS-touch를 활용하여 태양, 행성, 달 등 점 광원이 아닌 천체의 초점 조절 알고리즘을 개발할 생각이다. 앞서 언급한 것처럼 별은 점 광원이므로 FWHM이나 HFD를 이용하여 초점 조절하는 알고리즘이 개발되어 있으나 태양, 행성, 달 등 점 광원이 아닌 천체는 FWHM이나 HDF를 이용한 별의 자동 초점 조정 알고리즘을 바로 적용하기 힘들다. GS-touch를 이용하면 모터 포커서를 정밀하게 제어할 있으므로 태양, 행성, 달 등을 관측할 때 각각의 천체에 적합한 자동 초점 조절 알고리즘 개발하고 구현하는데 활용할 수 있을 것으로 생각된다. 
