\section{서론}
%\section{Introduction}


\subsection{지도 배경}

본 작품을 제작한 학생들은 2학년 학생으로 본교의 정규 교육과정에 포함되어 있는 자율연구 과목을 수강하게 되어 깉은 팀으로 구성되어 연구 주제를 정하는 과정을 거치게 되었다. 사사교육은 보다 전문화되고 개별화된 방법의 영재교육 방법이라고 할 수 있다. 따라서 영재아의 요
구와 특성에 보다 더 맞추어 교육시킬 수 있는 방법 중의 하나이다. 우리나라에서는 Research and Education (R\&E)라는 이름으로 사사교육이 도입되어 운영되고 있으며 그 형태는 다음과 같다. 
\begin{enumerate}
    \item 전국의 8개 과학영재학교(이하 영재고)에서는 정규 교육과정으로 운영되고 있으며 20 학점 내외의 연구학점에 포함되어 운영되는 경우가 있다. 
    \item 한국과학창의재단에서 전국의 영재고와 과학고등학교(이하 과학고)를 대상으로 과학영재 창의연구(R\&E)라는 프로그램으로 500여 개의 과제를 지원하여 운영하고 있다. 
    \item 한국과학창의재단에서 영재고와 과학고를 포함한 모든 고등학교를 대상으로 융합형 연구과제(STEAM R\&E)라는 프로그램으로 120개 연구팀을 지원하여 운영하고 있다. 
\end{enumerate}

R&E 활동의 유형은 크게 ‘실제 과학참여 연구’와 ‘자기주도적 프로젝트형 연구’로 구분될 수 있다 \cite{parkjw-2009-1}. 고등학생들이 R\&E를 시작하면서 겪는 가장 큰 어려움 중 하나는 바로 연구 주제 선정이다. 연구 주제 선정에 어려움을 겪는 이유는 여러가지가 있을 수 있는데, 체게이번에 전람회를 지도한 학생도 처음 R\&E 활동을 시작하는 단계에서 연구 주제, 연구 범위, 연구 방법 등을 잘 몰라서 시작을 하지 못하고 있었다. 

지도교사가 관심있거나 전문성이 확보된 연구 주제로 정하여 R\&E 활동을 지도하게 되면 자칫 학생의 창의력 계발을 저해한다고 생각할 수 있으나, 연구 주제를 정하지 못하고 시간을 허비하는 학생의 경우 학생의 경우 지도교사가 연구  주제를 제시하는 방법이 좋은 결과를 가져오는 경우도 있다. 

방파제 조적 구조에 의한 방파 효과를 분석, 비교하는 연구를 계획하던 중 모형 실험을 위하여 연안 모형과 이곳에 해파를 발생시키는 장치가 필요하게 되었다. 선행 연구를 조사해 보니 연안 등 다양한 조건에서 해양 현상의 모의 실험은 조파 수조를 이용하여 할 수 있다는 것을 알게 되었다\cite{chung2013}. 해양환경관리공단(KOEM)은 해양환경개발교육원 내에 세계 최초로 인공해안이 설치된 조파 수조를 설치하고 발명 특허를 내기도 하였다 (등록번호 제10-0978231호). 또, KITECH 해양로봇센터, CIIZ, KIOST 한국해양과학기술원 등 다양한 기관에서는 대규모 조파 시설을 구비하여 타 기관이나 업체에서 시험을 의뢰받기도 한다. 본교에서도 해양 관련 연구가 자주 이뤄지기에 다양한 실험을 시행하기 위해서 조파 수조가 필요하지만 이는 시중에서 구하기 어려우며 상당히 고가이다. 이에 여러 해안 환경 모형 실험을 할 수 있는 조파 수조(wave flume)와 맞춤형 파를 제작할 수 있는 조파기(wave maker)를 제작하기로 하였다. 


\subsection{제작 목적}

본 조파 수조는 파도를 발생시카거나 이를 이용한 모형 실험을 하기 위하여 개발되었다. 조파 수조는 조파기, 소파기, 파고계, 연안 모형 등의 관련 시설을 포함하여 아래와 같은 모형 실험을 할 수 있도록 하는데 주된 목적이 있다.

\begin{enumerate}
    \item 쓰나미에 의한 피해를 알아보는 연안의 구조물 모형 실험
    \item 다양한 파력발전 모형 실험
    \item 방파제의 조적구조 효율 비교 실험
    \item 선박, 부표 등의 안정성 비교 실험
\end{enumerate}

본 연구에서는 2차원 조파수조를 제작하고 이를 이용하여 다양한 실험을 할 수 있도록 조파기의 성능을 검증하였다. 


\subsection{지도 개요}

본 연구에서는 천체 망원경의 포커서를 손으로 조절할 경우에 생기는 진동으로 생기는 문제점을 해결하기 위하여 포커서에 스테핑 모터를 장착한 모터 포커서를 정밀하게 제어할 수 있는 모터 포커서 컨트롤러를 제작하여 실제 관측에 사용하는 것을 목적으로 한다. 제작한 모터 포커서 컨트롤러는 GS-touch (Gyeonggi Science touch)로 명명하였다. GS-touch는 스테핑 모터를 구동할 수 있고 컴퓨터로 제어할 수 있도록 설계하였다. 또한 전용 ASCOM(Astronomy Common Object Model) dirver를 개발하여 ASCOM을 지원하는 천문 소프트웨어를 이용하여 쉽게 사용할 수 있도록 하였다. 

하나의 하드웨어를 개발하고 이를 동작하도록 펌웨어와 ASCOM 드라이버까지 제작하기 위해서 학생이 필요한 소프트웨어 같은 툴을 익히는데 많은 시간이 소요되었다. R\&E 활동을 통해 지도한 내용을 Table \ref{table:teaching}에 나타내었다. 학생은 영재의 속성인 과제 집착력을 보이면서 빠르게 툴의 사용법을 익혀 나갔고, 결국 연구를 완료하는 끈기 있는 모습을 보여 주었다. 

\begin{table}[htbp]
	\caption{지도 개요}
	\begin{tabular}{l|l}
		\toprule[1pt]
		절차                  & 지도 내용                                                                                                                            \\
		\toprule[1pt]
		1. 관련 자료 검색         & - 선행 연구, 천체 관측에 대한 배경 지식 등을 학습하도록 지도                                                                                             \\
		\midrule[0.6pt]
		2. 연구 문제 구체화        & \begin{tabular}[c]{@{}l@{}}- 기존 제품의 동작 원리 파악\\ - 개발할 모터 포커서 제어 장치에 필요한 기능결정\end{tabular}                                         \\
		\midrule[0.6pt]
		3. 하드웨어 제작 방법 지도    & - 회로 설계 방법, 납땜 방법, 엔클로저 제작 방법 지도 지도                                                                                                             \\
		\midrule[0.6pt]
		4. 펌웨어 개발 지도        & - 아두이노 통합 개발 환경을 이용한 펌웨어 개발방법 지도                                                                                                 \\
		\midrule[0.6pt]
		5. ASCOM 드라이버 개발 지도 & \begin{tabular}[c]{@{}l@{}}- 펌웨어와 PC 간 통신을 하기 위한 규약 설정\\ - Microsoft visula studio 2017을 이용한 ASCOM 드라이버 \\ 및 응용프로그램 개발\end{tabular} \\
		\midrule[0.6pt]
		6. 실제 동작 확인         & - 천체 망원경의 모터 포커서에 장착하여 실제 동작 확인                                                                                                  \\
		\midrule[0.6pt]
		7. 보고서 작성 지도        & \begin{tabular}[c]{@{}l@{}}- Latex를 사용법 지도\\ - APA style 보고서 작성법 지도\end{tabular}\\
		\bottomrule[1pt]                                                 
	\end{tabular}
	\label{table:teaching}
\end{table}

본 연구의 또 다른 목적은 관심있는 사람들은 직접 부품을 구입하여 제작할 수 있도록 제작 과정과 노하우를 공개하여 아이디어 나눔을 실천하는 것도 포함된다.



\subsection{연구 문제}

2차원 조파 수조 제작 지도와 관련된 본 연구에서 다루는 연구 문제는 다음과 같다. 

1. 조파 수조에 물이 들어 있는 상태에서 큰 힘으로 물을 아두이노보다 성능이 좋은 MCU를 이용하여  나노를 기반으로 2상 바이폴라 스테핑 모터 드라이버, OLED 디스플레이어, 온습도 센서 등이 연결된 모터 포커서 컨트롤러인 GS-touch의 하드웨어를 제작할 수 있는가?

2. 2차원 미니 조파 수조에서 조파기로 여러 종류의 해파를 누구나 쉽게 만들 수 있도록 수 있는가?

3. 2차원 미니 조파 수조로 우리나라의 동해, 황해, 남해 등 조건이 다른 여러 연안 환경에서 일어날 수 있는 모형 실험을 수행할 수 있도록 구현할 수 있는가?
