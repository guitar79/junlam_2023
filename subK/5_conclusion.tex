\section{결론}

본 연구에서는 연안 환경 모형 실험을 위한 2차원 조파 수조를 제작하였다. 조파수조는 조파기, 소파기, 연안 모형 등으로 구성되며 수조는 길이 $2,000~\mathrm{mm}$, 높이 $400~\mathrm{mm}$, 폭 $300~\mathrm{mm}$의 모듈 3개로 이루어진 $6,000~\mathrm{mm}$ 길이의 구조체이다. 규격에 맞는 모듈을 추가 제작하여 길이 연장이 가능하다. 조파기는 $80~\mathrm{cm}$ 길이의 리니어 액츄에이터에 아크릴 조파판이 장착되어 움직이는 구동부와 틴지 보드와 모터 드라이버로 회로를 조종하는 제어부가 있다. 조파 수조 내부에서 조파판이 물을 밀어내며 파를 생성하며 수조 말단에는 소파기를 두어 파를 소멸한다. 소파기는 다공성 구조를 가진 플라스틱 바구니를 5층으로 쌓았으며 안에 철 수세미를 채워넣어 효율적으로 파를 소멸시킬 수 있도록 하였다. 조파기의 반대편에는 연안 모형이 있으며 현재 길이 $2,000~\mathrm{mm}$, 높이 $200~\mathrm{mm}$로 기울기가 1/10인 경사로를 설치하였고 탈부착이 가능하여 추후 다른 실험을 진행할 때 제거할 수 있다.

조파기는 다양한 규칙파를 만들 수 있으며 코드 상에서 스텝 모터의 각 위치를 직접 대입하여 변화시킨다. 스텝 모터가 $\sin$형 각 변위를 따를 때 진폭과 진동수를 조작할 수 있으며 생성파의 진동수는 입력값과 같지만 파고는 적당한 Calibration을 통하여 찾아야 한다. 본 연구에서는 조파기가 생성한 파의 $H/S$와 이론적인 $H/S$에 대한 근사적인 식을 제시하였으며 조파판과 벽 사이로 물이 새는 등 여러 오차 요인이 존재하여 실험적으로 주어진 파고와 스트로크에 대한 코드의 매개변수 값을 정해야 한다. 구축된 시스템은 여러 기기에 응용할 수 있으며 사용하는 리니어 액츄에이터에 따라서 섬세한 파도 혹은 강력한 파도를 생성할 수 있다.

본 연구에서 제작한 조파 수조는 연안 환경 시뮬레이션이 필요한 연구에 널리 사용될 수 있다. 쓰나미에 의한 피해를 알아보는 연안의 구조물 모형 실험, 파력발전 모형의 효율 비교, 방파제 조적구조의 성능 비교나 선박, 부표 등의 안정성 비교 등의 실험에서 쓰일 수 있다. 현재 본교에서 진동수주형 파력 발전과 관련된 연구가 진행 중이며 조파 수조가 사용될 예정이다.