\section{서론}
%\section{Introduction}


\subsection{제작 동기}

방파제 조적 구조에 의한 방파 효과를 분석, 비교하는 연구를 계획하던 중 모형 실험을 위하여 연안 모형과 이곳에 해파를 발생시키는 장치가 필요하게 되었다. 선행 연구를 조사해 보니 연안 등 다양한 조건에서 해양 현상의 모의 실험은 조파 수조를 이용하여 할 수 있다는 것을 알게 되었다\cite{chung2013}. 해양환경관리공단(KOEM)은 해양환경개발교육원 내에 세계 최초로 인공해안이 설치된 조파 수조를 설치하고 발명 특허를 내기도 하였다 (등록번호 제10-0978231호). 또, KITECH 해양로봇센터, CIIZ, KIOST 한국해양과학기술원 등 다양한 기관에서는 대규모 조파 시설을 구비하여 타 기관이나 업체에서 시험을 의뢰받기도 한다. 본교에서도 해양 관련 연구가 자주 이뤄지기에 다양한 실험을 시행하기 위해서 조파 수조가 필요하지만 이는 시중에서 구하기 어려우며 상당히 고가이다. 이에 여러 해안 환경 모형 실험을 할 수 있는 조파 수조(wave flume)와 맞춤형 파를 제작할 수 있는 조파기(wave maker)를 제작하기로 하였다. 


\subsection{제작 목적}

본 조파 수조는 파도를 발생시카거나 이를 이용한 모형 실험을 하기 위하여 개발되었다. 조파 수조는 조파기, 소파기, 파고계, 연안 모형 등의 관련 시설을 포함하여 아래와 같은 모형 실험을 할 수 있도록 하는데 주된 목적이 있다.

\begin{enumerate}
    \item 쓰나미에 의한 피해를 알아보는 연안의 구조물 모형 실험
    \item 다양한 파력발전 모형 실험
    \item 방파제의 조적구조 효율 비교 실험
    \item 선박, 부표 등의 안정성 비교 실험
\end{enumerate}

본 연구에서는 2차원 조파수조를 제작하고 이를 이용하여 다양한 실험을 할 수 있도록 조파기의 성능을 검증하였다. 
