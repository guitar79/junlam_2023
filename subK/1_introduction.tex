\section{서론}
%\section{Introduction}


\subsection{제작 동기}

방파제 조적 구조에 의한 방파 효과를 분석, 비교하는 연구를 계획하던 중 모형 실험을 위하여 연안 모형과 해파를 발생시키는 장치가 필요하게 되었다. 선행 연구에 따르면 연안 등 다양한 조건에서 해양 현상의 모의 실험은 조파 수조를 이용하여 할 수 있다\cite{chung2013}. 해양환경관리공단(KOEM)은 해양환경개발교육원 내에 세계 최초로 인공해안이 설치된 조파 수조를 설치하고 발명 특허를 내기도 하였다 (등록번호 제10-0978231호). KITECH 해양로봇센터, CIIZ, KIOST 한국해양과학기술원 등 다양한 기관에서는 대규모 조파 시설을 구비하여 타 기관이나 업체에서 시험을 의뢰받기도 한다. 조파 수조는 연안 환경을 재현하며 다양한 실험을 가능케 한다.

현재 기후위기가 시간이 지남에 따라 더욱 심화되어가면서 연안 지형이 가장 큰 위협을 받고 있다\cite{bini2021climate}. 쓰나미, 태풍 등의 다양한 재해에 대해 연안 모형에서 대처할 공법이 필요하나, 연안 지형에 방파제나 경사로 등을 직접 설치하여 시험하기에는 공법이 실패할 경우에 지출이 과다하므로, 이를 축소하여 모형으로 먼저 실험하여 시행 여부를 결정해야 한다. 2차원 조파 수조를 제작하여 경사로 등을 설치하면 연안 환경을 조성한 후 방파제 모형 등의 방파 성능을 시험할 수 있다. 또한 신재생에너지 분야에서, 파력 발전을 위한 발전체를 작게 제작하여 조파 수조에 설치해 발전체의 효율을 시험할 수 있다.

본교에서도 해양 관련 연구가 자주 이뤄지기에 다양한 실험을 시행하기 위해서 조파 수조가 필요하지만 이는 시중에서 구하기 어려우며 상당히 고가이다. 이에 여러 해안 환경 모형 실험을 할 수 있는 조파 수조(wave flume)와 맞춤형 파를 제작할 수 있는 조파기(wave maker)를 제작하기로 하였다. 


\subsection{제작 목적}

본 연구에서 제작한 조파 수조는 파도를 발생시키며 연안의 환경을 조형하여 모형 실험을 하기 위해 개발되었다. 조파 수조는 조파기, 소파기, 파고계, 연안 모형 등의 관련 시설을 포함하며 아래와 같은 모형 실험을 할 수 있도록 하는데 주된 목적이 있다.

\begin{itemize}
    \item 쓰나미에 의한 피해를 알아보는 연안의 구조물 모형 실험
    \item 다양한 파력발전 모형 실험
    \item 방파제의 조적구조 성능 비교 실험
    \item 선박, 부표 등의 안정성 비교 실험
\end{itemize}

2차원 조파수조를 제작하고 이를 이용하여 다양한 실험을 할 수 있도록 조파기의 성능을 검증하였다. 파는 규칙파를 우선적으로 생성한다. 즉, 다음과 같은 연구 과제를 설정하였다.

\begin{enumerate}
    \item 규칙파를 생성할 수 있는 조파기를 제작할 수 있는가?
    \item 매개변수로 파의 개형을 조절할 수 있는가?
    \item 조파기 외의 연안 환경 재현을 위한 다른 구조물을 제작하였는가?
\end{enumerate}