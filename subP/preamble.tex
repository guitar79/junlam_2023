\documentclass{junlam_report}
% 아래의 함수를 사용하면 이미지 파일들을 같은 디렉토리 내에 images 라는 이름을 가진 폴더를 생성한 후, 그 폴더 안에 넣어 사용할 수 있습니다.
% 사용하고자 한다면 주석을 푸십시오.
\graphicspath{{images/}}
% 이곳에 필요한 별도의 패키지들을 적어넣으시오.
%\usepackage{...}
\usepackage{verbatim} % for commment, verbatim environment
\usepackage{spverbatim} % automatic linebreak verbatim environment
%\usepacakge{indentfirst}
\usepackage{tikz}
%\tikzset{
%	image label/.style={
%		every node/.style={
			%fill=black,
			%text=white,
%			font=\sffamily\scriptsize,
%			anchor=south west,
%			xshift=0,
%			yshift=0,
%			at={(0,0)}
%		}
%	}
%}
\usepackage{amsmath}
\usepackage{amsfonts}
\usepackage{amssymb}
\usepackage{algorithm}
\usepackage{algpseudocode}
\usepackage{float}
\usepackage{graphicx}
\usepackage{tabularx}
\usepackage{multirow}
\usepackage{multicol}
% \setlength{\columnseprule}{0.5pt}
% \def\columnseprulecolor{\color{black}}


\usepackage{booktabs}
\usepackage{longtable}
\usepackage{gensymb}
\usepackage{wrapfig}
%\usepackage{subcaption}
%\usepackage{floatrow}
%\usepackage{pict2e}
%\usepackage[backend=biber,style=authoryear]{biblatex}
%\usepackage{biblatex}
\usepackage{pgfplots}
\pgfplotsset{
	compat=newest,
	label style={font=\sffamily\scriptsize},
	ticklabel style={font=\sffamily\scriptsize},
	legend style={font=\sffamily\tiny},
	major tick length=0.1cm,
	minor tick length=0.05cm,
	every x tick/.style={black},
}

\usetikzlibrary{shapes}
\usetikzlibrary{plotmarks}
\usepackage{listings}
\usepackage{hologo}
\usepackage{makecell}
\usepackage{color}
\lstset{
	basicstyle=\small\ttfamily,
	columns=flexible,
	breaklines=true
}

\citation
\bibdata

%: ----------------------------------------------------------------------
%:               보고서 정보를 입력하시오
% ----------------------------------------------------------------------
% 아래와 같은 command를 만들면 길이가 긴 용어를 간편하게 사용할 수 있습니다. 단, 이미 지정된 함수명들은 새로운 함수명으로 사용할 수 없습니다.
% 연, 월, 일은 보고서 제출 날짜에 맞게 수정하십시오.

% 출품자, 지도교사는 기재하지 않습니다.

% 출품 번호
\summitnumber{지-6} % 

% 제출일과 과학전람회 회수
\summitdate{2023}{6}{1} % (연, 월, 일)
\junlam{69} % 과학전람회 회수 = 연도 - 1954 (ex. 2020년은 66회 대회)

% 출품 분야
\entryfield{학생} % 학생 / 교원

% 출품 부문
\entrysection{지구 및 환경} % 물리 / 화학 / 생물 / 산업 및 에너지 / 지구 및 환경

% 제목
% \title{확장 가능한 2차원 미니 조파 수조 제작 및 이를 활용한 연안 환경 모형 실험} % 제목 개행 시 \linebreak 사용. \\나 \newline 은 안됨.
\title{연안 환경 모형 실험을 위한 $2$차원 조파 수조 제작} % 제목 개행 

\newtheorem{definition}{정의}

