% * 필요한 사진

% 1. 조파기 종류 별 사진 (2.2)
% 2. 조파수조 6m가 전체적으로 나온 사진 (3.1) //찍
% 3. 수세미를 담은 바구니 사진 (3.1.1)
% 4. 우리가 만든 경사로 (3.1.2) 
% 5. 조파기가 수조 위에 올려진 사진 (3.2.1)
% 6. 파고계 사진 (3.3)

% * 있으면 좋을 거 같으나 굳이 없어도 될 사진
% 1. 모듈과 모듈 사이 밀착한 사진 (3.1)
% 2. 조파기 설계도(새로 그리기) (3.2)
% 3. 수조 내부에 판이 벽에 밀착된 사진 (3.2.1)


% * 필요한 논문
% 1. 소파장치 관련 논문 (3.1.1)
% 2. 경사로 규격 관련 논문 (3.1.2)

- 조파기 사진 넣기

@article{ART002413750,
author={오정근 and 김주열 and 김효철 and 권종오 and 류재문},
title={수직 평판 요소의 수중동요 근사해와 설계 적용},
journal={대한조선학회 논문집},
issn={1225-1143},
year={2018},
volume={55},
number={6},
pages={527-534},
doi={10.3744/SNAK.2018.55.6.527}
}

@article{ART002291815,
author={권종오 and 김효철 and 류재문 and 오정근},
title={조파판 수중운동의 근사해석과 조파기 설계에 응용},
journal={대한조선학회 논문집},
issn={1225-1143},
year={2017},
volume={54},
number={6},
pages={461-469},
doi={10.3744/SNAK.2017.54.6.461}
}

@article{ART002785404,
author={김효철 and 오정근 and 류재문 and 이신형 and 김재헌},
title={다기능 조파기의 조파 운동과 발생 파형},
journal={대한조선학회 논문집},
issn={1225-1143},
year={2021},
volume={58},
number={6},
pages={339-347},
doi={10.3744/SNAK.2021.58.6.339}
}

@inproceedings{Oh-2018,
  title={한국해양과학기술원 2 차원 조파수조 구축},
  author={오상호 and 장세철},
  booktitle={한국연안방재학회 연례학술대회},
  pages={1--1},
  year={2018},
  organization={한국연안방재학회}
}
@article{Jeong2021-2D,
  title={이차원 조파수조에서 복합형 타공판의 소파 성능에 관한 실험적 연구},
  author={정현철 and 구원철},
  journal={Journal of Ocean Engineering and Technology},
  volume={35},
  number={2},
  pages={113--120},
  year={2021}
}

@inproceedings{Oh-2019,
  title={데이터모델링 기법을 이용한 2 차원 조파수조 내 천해 파고 추정},
  author={오정은 and 오상호},
  booktitle={한국해안해양공학회 학술발표논문집},
  pages={28--28},
  year={2019},
  organization={한국해안}
}

@article{KJO2017wave,
  title={조파판 수중운동의 근사해석과 조파기 설계에 응용},
  author={권종오 and 김효철 and 류재문 and 오정근},
  journal={대한조선학회 논문집},
  volume={54},
  number={6},
  pages={461--469},
  year={2017}
}

@article{OJK2018vertical,
  title={수직 평판 요소의 수중동요 근사해와 설계 적용},
  author={오정근 and 김주열 and 김효철 and 권종오 and 류재문},
  journal={대한조선학회 논문집},
  volume={55},
  number={6},
  pages={527--534},
  year={2018}
}